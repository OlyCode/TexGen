\documentclass[12pt]{article} 
\usepackage[utf8]{inputenc} % set input encoding
%%% Examples of Article customizations
% These packages are optional, depending whether you want the features 
% they provide.
% See the LaTeX Companion or other references for full information.
%%% PAGE DIMENSIONS
\usepackage{geometry} % to change the page dimensions
 \geometry{letterpaper} % or letterpaper (US) or a5paper or....
 \geometry{margin=1in} % change the margins to 1 inch all round
 \geometry{portrait} % set up the page for landscape
\setlength\parindent{0pt}
%\usepackage{graphicx} % support the \includegraphics command and options
% \usepackage[parfill]{parskip} % Activate to begin paragraphs with an 
% empty line rather than an indent
%%% PACKAGES
\usepackage{pgfplots}
\pgfplotsset{compat=1.15}

%\usepackage{booktabs} % for much better looking tables
%\usepackage{array} % for better arrays (eg matrices) in maths
%\usepackage{paralist} % very flexible & customisable lists 
%                     %(eg. enumerate/itemize, etc.)
%\usepackage{verbatim} % adds environment for commenting out blocks 
%                     % of text & for better verbatim
%\usepackage{subfig} % make it possible to include more than one captioned 
%                   % figure/table
% These packages are all incorporated in the memoir class to one degree 
% or another...

\begin{document}
%\section{First section}
%\subsection{A subsection}
\begin{flushright}
Name: \underline{\hspace{2.5 in}}
\end{flushright}
\begin{flushright}
Worksheet ID: %\tgID{}
\end{flushright}
\section*{Linear and Exponential Functions (M.3A)}

\begin{tikzpicture}
  \begin{axis}[ 
	xmin=-5,   xmax=5,
	ymin=-5,   ymax=5,
	xtick={-5,...,5},
	ytick={-5,...,5},
    xlabel=$x$,
    ylabel={$f(x) = x^2 - x +4$}
  ] 
    \addplot {x^2 - x +4}; 
  \end{axis}
\end{tikzpicture}

Use the graph above to answer the following quesitons:

1. Find $f(1)$.

2. Find $x$ when $f(x) = 2$. 

\hline

%\vspace{1 in}

\begin{tabular}{|c|c|}
\hline 
$x$ & $g(x)$ \tabularnewline
\hline 
\hline 
$3$ & $5$ \tabularnewline
\hline 
$3$ & $5$ \tabularnewline
\hline 
$3$ & $5$ \tabularnewline
\hline 
$3$ & $5$ \tabularnewline
\hline 
$3$ & $5$ \tabularnewline
\hline 
$3$ & $5$ \tabularnewline
\hline 
\end{tabular}

Use the table above to answer the following quesitons:

3. Find $g(4)$.

4. Find $g(x)$ when $x=7$.

5. Find $x$ when $g(x) = 3$.

\hline

\end{document}

